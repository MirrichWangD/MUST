離線手寫簽名驗證是生物特徵技術的應用, 其根據用戶提供的手寫簽名与存儲的参考簽名對比以驗證用戶身份, 被廣泛用於安全認證, 金融交易等安全領域. 學術研究中, 學者們定義了作者獨立和作者依賴任務以評估離綫手寫簽名驗證模型的泛化和判斷性能.

深度學習的視覺模型在離綫手寫簽名驗證領域中取得了顯著進展, 但是模型對高像素的手寫簽名圖像特徵學習能力具有一定的缺陷, 并且注意力機制的Transformer具有訓練過程模型收斂速度慢的缺點. 為了在一定程度上解決該問題, 本工作基於Transformer提出OSVTF端到端模型結構, 在組合特徵的基礎上增加多尺度融合特徵, 并且對部分子網絡進行了加速訓練收斂的優化. 

在實驗部分, OSVTF模型在BHSig-B 80/20數據集的作者獨立和作者依賴任務中EER為4.54\%和2.13\%, 在BHSig-H 100/60數據集的EER為3.90\%和2.68\%, 達到了日常生產使用的水準. 在消融實驗部分, OSVTF在CEDAR 50/5數據集的作者獨立任務中EER為4.75\%, 證實加速訓練收斂優化的有效性. 整體實驗結果證實在跨數據集的場景下, OSVTF相較過去的離綫簽名驗證模型在性能上取得了一定程度的提升, 在此基礎上具有更優秀手寫簽名圖像特徵學習能力和更快的訓練收斂速度.



% 本文研究離散事件繫統的監督控制問題。
% 在使用該範本中有任何問題請聯繫覃濤
% \href{mailto:zhwli@ieee.org}{zhwli@ieee.org}。
% 澳門科技大學系統工程研究所感謝覃濤對設計此範本的貢獻。


% \medskip\medskip

% The template can be used in online and offline ways. For the former (highly recommended), 
% Overleaf (\url{https://www.overleaf.com}) is a collaborative cloud-based LaTeX editor used for writing, editing and publishing scientific documents, which is much easy to use and friendly. In overleaf, the compiling command is \textcolor{blue}{XeLatex}.

% For the latter, one can use Texstudio, which is a very popular yet free software package (\url{https://www.texstudio.org/}). When using Texstudio, the compiling command is \textcolor{blue}{XeLatex}. To make Texstudio work, one need to first install \textcolor{blue}{Miktex}, see \url{https://miktex.org/}. We happen to find, rather rarely, that a successful compiling may depend on the version of Texstudio. In any case, we recommend the latest version of Texstudio.




