離線手寫簽名驗證是生物特徵技術的一個應用场景, 其根據用戶提供的手寫簽名与資料庫中該用戶存儲的手寫簽名進行對比以驗證用戶身份, 在日常生活中被廣泛用於安全認證, 金融交易等安全領域. 學術研究中, 離線手寫簽名驗證定義為作者獨立和作者依賴任務. 第一種任務是將手寫簽名與對應作者的參考簽名進行對比驗證. 第二種任務是在作者依賴任務的基礎上使用獨立的作者分類器判斷輸入簽名是否是偽造的.

本工作内容如下: 1. 提出圖像多尺度融合特徵的OSVTF端到端模型結構,初步驗證模型架構性能. 2. 針對多尺度特徵融合和模型優化部分, 進行消融實驗以證實模型子網絡組合方案的性能. 3. 針對作者依賴任務,將對比支持向量機或全局平均池化作爲分類器的OSVTF性能.

在實驗部分, OSVTF模型在BHSig-B 80/20數據集的作者獨立和作者依賴任務中EER為4.54\%和2.13\%, 在BHSig-H 100/60數據集的EER為3.90\%和2.68\%. 在消融實驗部分, OSVTF在CEDAR 50/5數據集的作者獨立任務中EER為4.75\%. 結果證實采取了多尺度融合特徵的OSVTF能夠加强對僞造簽名的敏感部分特徵學習, 部分優化將加快模型訓練過程的收斂速度.


% 本文研究離散事件繫統的監督控制問題。
% 在使用該範本中有任何問題請聯繫覃濤
% \href{mailto:zhwli@ieee.org}{zhwli@ieee.org}。
% 澳門科技大學系統工程研究所感謝覃濤對設計此範本的貢獻。


% \medskip\medskip

% The template can be used in online and offline ways. For the former (highly recommended), 
% Overleaf (\url{https://www.overleaf.com}) is a collaborative cloud-based LaTeX editor used for writing, editing and publishing scientific documents, which is much easy to use and friendly. In overleaf, the compiling command is \textcolor{blue}{XeLatex}.

% For the latter, one can use Texstudio, which is a very popular yet free software package (\url{https://www.texstudio.org/}). When using Texstudio, the compiling command is \textcolor{blue}{XeLatex}. To make Texstudio work, one need to first install \textcolor{blue}{Miktex}, see \url{https://miktex.org/}. We happen to find, rather rarely, that a successful compiling may depend on the version of Texstudio. In any case, we recommend the latest version of Texstudio.




