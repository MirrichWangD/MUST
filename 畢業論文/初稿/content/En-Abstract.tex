Offline handwritten signature verification is an application of biometric technology, which is widely used in security verification, financial transactions, and other security domains to verify the identity of a user based on the comparison of the user-supplied handwritten signature with the stored reference signature. In academic research, writer-independent and writer-dependent tasks have been defined to evaluate the generalization and judgmental performance of offline handwritten signature verification models.

Deep learning visual models have made significant progress in the field of offline handwritten signature verification, but the models have some shortcomings in their ability to learn the features of high pixel handwritten signature images, and the Transformer with attention mechanism has the disadvantage of slow convergence of the model during the training process. In order to solve the problem to a certain extent, this work proposes an end-to-end model structure of OSVTF based on Transformer, which adds multi-scale fusion features on the basis of combined features, and optimizes some sub-networks to accelerate the training convergence. 

In the experimental part, the OSVTF model achieves EERs of 4.54\% and 2.13\% for writer-independent and writer-dependent tasks on the BHSig-B 80/20 dataset, and 3.90\% and 2.68\% on the BHSig-H 100/60 dataset, which is up to the level of daily production use. In the ablation experiment part, OSVTF has an EER of 4.75\% in the writer-independent task on the CEDAR 50/5 dataset, which confirms the effectiveness of the accelerated training convergence optimization. The overall experimental results confirm that OSVTF achieves a certain degree of performance improvement over previous offline signature verification models in cross-dataset scenarios, based on which it has better handwritten signature image feature learning capability and faster training convergence speed.



% This research deals with the supervisory control problem of discrete event systems.


% Do not say something like ``This paper''.

% (Use singular keywords. Keywords are separated by commas or semicolons, and there is often a period at the end.)